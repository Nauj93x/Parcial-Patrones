% Documento LaTeX del proyecto Flyweight con sección para imagen de salida de consola
\documentclass[11pt,a4paper]{article}
\usepackage[utf8]{inputenc}
\usepackage[T1]{fontenc}
\usepackage[spanish]{babel}
\usepackage{geometry}
\usepackage{hyperref}
\usepackage{listings}
\usepackage{xcolor}
\usepackage{graphicx}
\usepackage{parskip}
\geometry{margin=2.5cm}

\definecolor{codegray}{rgb}{0.95,0.95,0.95}
\definecolor{keyword}{rgb}{0.36,0.1,0.6}
\lstset{
  backgroundcolor=\color{codegray},
  keywordstyle=\color{keyword}\bfseries,
  basicstyle=\ttfamily\small,
  breaklines=true,
  frame=single,
  numbers=left,
  numberstyle=\tiny,
  showstringspaces=false,
  commentstyle=\itshape\color{gray},
  captionpos=b
}

\title{Análisis detallado del código — Patrón Flyweight con persistencia en Supabase}
\author{Proyecto: Flyweight PARCIAL}
\date{\today}

\begin{document}
\maketitle

\begin{abstract}
Documento técnico orientado a sustentación: descripción puntual de las clases clave, métodos de mayor interés, contratos (precondiciones/postcondiciones), y notas de implementación y pruebas. Incluye una sección para insertar una imagen con la salida de consola que puedas capturar y guardar como \texttt{console_output_example.png} en el mismo directorio del proyecto.
\end{abstract}

% (Contenido resumido — puedes insertar aquí el resto del texto que generamos previamente)

\section{Salida de consola (imagen)}
A continuación se muestra un ejemplo de cómo incluir una imagen con la salida de consola en tu documento. Guarda la captura de la consola como \texttt{console_output_example.png} en la carpeta del proyecto (por ejemplo en el mismo directorio que este .tex o en \texttt{docs/}) y compila en Overleaf o localmente.

\begin{figure}[ht]
  \centering
  \includegraphics[width=0.95\textwidth]{console_output_example.png}
  \caption{Ejemplo de salida en consola: arranque, menú y mensajes de evicción/persistencia. Guarda tu imagen con el nombre \texttt{console_output_example.png} para que se muestre aquí.}
  \label{fig:console}
\end{figure}

\section{Instrucciones rápidas}
\begin{itemize}
  \item Captura la salida de la consola durante la demo (por ejemplo con la herramienta de captura de tu sistema o redirigiendo la salida a un archivo y sacando una imagen).
  \item Nombra la imagen exactamente \texttt{console_output_example.png} y súbela a Overleaf en el mismo proyecto (o cópiala en la carpeta \texttt{docs/} si compilas localmente).
  \item Compila el documento; la figura se insertará en la sección de salida de consola.
\end{itemize}

\bigskip
\textbf{Nota:} si prefieres otro formato o nombre de archivo, actualiza el argumento de \texttt{\includegraphics\{...\}}.

\end{document}
